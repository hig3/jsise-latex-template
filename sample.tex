\documentclass[a4j]{article}

% ここに,使用するパッケージを列挙する.
\usepackage{jsise}
%\usepackage{uarial}
\usepackage{url}
\usepackage [dvipdfmx] {graphicx}
%\usepackage{fancybox}

\begin{document}
% 和文タイトル
\title{研究会報告書書式の見本}

% 和文著者・所属
\author{
  	研究会委員会研究報告書式担当\afil{*1} \quad \quad %
        研究会委員会\afil{*1}\\
	\afil{*1} 教育システム情報学会}

% 英文タイトル
\etitle{Format of the Research Report}

% 英文著者・所属
\eauthor{%
	Sig Committee Research Report Format Group\afil{*1} \quad \quad %
        Sig Committee\afil{*1}\\
	\afil{*1} Japanese Society for Information and Systems in Education}

\abstract{
\msnormalsize
Information processing is the change (processing) of information in any manner detectable by an observer. As such, it is a process which describes everything which happens (changes) in the universe, from the falling of a rock (a change in position) to the printing of a text file from a digital computer system. In the latter case, an information processor is changing the form of presentation of that text file. Information processing may more specifically be defined in terms used by Claude E.
}

\keywords{研究会報告,書式,執筆要領}

\maketitle
\thispagestyle {empty}

\section{はじめに}
\msnormalsize
従来,研究会報告原稿は学会誌執筆要領に従って作成することとされていた.しかし,学会誌投稿原稿はそのまま印刷することを考慮していないため,フォントサイズの指定がない,マージンの指定がないなど,一部詳細が定められておらず,あいまいな部分があった.そこで,今回,学会誌執筆要領を参考にして,不足している情報を補い,執筆要領を定めた.

なお,この文章は研究会報告の書式例として用意したものである.研究報告作成に当たっては,学会ホームページに掲載している「研究報告執筆要領」を参照されたい.

\section{研究会報告の書式}
\subsection{書式の概要}

\msnormalsize
研究会報告の原稿には,A4用紙を使用する.マージンは,上下左右とも15mmとし,タイトル部分は1段組,本文は2段組とする.

タイトル部分には,日本語表題,著者名,所属,英語表題,著者名,所属,それに続けて,英文または和文の概要,さらに日本語キーワードを記述する.

\subsection{本文の書式}

\msnormalsize
本文は,章,節,項,等の見出しをつけて読みやすく構成する.見た目を学会誌とそろえるため,本文の書式は以下の様にする.

\subsubsection{段組構成}

\msnormalsize
本文は2段組として,1ページ41行,一段の横の長さを24文字とする.段間は10mmである.

章,節の区切りを見やすくするために,章タイトルの前後に空白行を1行挿入する.節タイトルの上には空白行を1行挿入する.章タイトルと節タイトルが連続する場合には,この間に1行の空白を入れる.

\subsubsection{フォント}

\msnormalsize
章,節タイトルはゴシック系のフォント,それ以外の本文は明朝系のフォントを使用する.章タイトルの文字サイズのみ12ptとして,それ以外の本文は10ptとする.

\subsubsection{制限枚数}

\msnormalsize
2ページ以上,8ページ以内とすること.

\section{おわりに}

\msnormalsize
この書式で書かれた原稿からオンライン報告書を作成する.図表の作成にあたってはカラーの図や写真も使うことができるが,不鮮明にならないよう留意されたい.
\newpage
\msnormalsize
\noindent%
01あいうえおあいうえおあいうえおあいうえおあい
02あいうえおあいうえおあいうえおあいうえおあい
03あいうえおあいうえおあいうえおあいうえおあい
04あいうえおあいうえおあいうえおあいうえおあい
05あいうえおあいうえおあいうえおあいうえおあい
06あいうえおあいうえおあいうえおあいうえおあい
07あいうえおあいうえおあいうえおあいうえおあい
08あいうえおあいうえおあいうえおあいうえおあい
09あいうえおあいうえおあいうえおあいうえおあい
10あいうえおあいうえおあいうえおあいうえおあい
11あいうえおあいうえおあいうえおあいうえおあい
12あいうえおあいうえおあいうえおあいうえおあい
13あいうえおあいうえおあいうえおあいうえおあい
14あいうえおあいうえおあいうえおあいうえおあい
15あいうえおあいうえおあいうえおあいうえおあい
16あいうえおあいうえおあいうえおあいうえおあい
17あいうえおあいうえおあいうえおあいうえおあい
18あいうえおあいうえおあいうえおあいうえおあい
19あいうえおあいうえおあいうえおあいうえおあい
20あいうえおあいうえおあいうえおあいうえおあい
21あいうえおあいうえおあいうえおあいうえおあい
22あいうえおあいうえおあいうえおあいうえおあい
23あいうえおあいうえおあいうえおあいうえおあい
24あいうえおあいうえおあいうえおあいうえおあい
25あいうえおあいうえおあいうえおあいうえおあい
26あいうえおあいうえおあいうえおあいうえおあい
27あいうえおあいうえおあいうえおあいうえおあい
28あいうえおあいうえおあいうえおあいうえおあい
29あいうえおあいうえおあいうえおあいうえおあい
30あいうえおあいうえおあいうえおあいうえおあい
31あいうえおあいうえおあいうえおあいうえおあい
32あいうえおあいうえおあいうえおあいうえおあい
33あいうえおあいうえおあいうえおあいうえおあい
34あいうえおあいうえおあいうえおあいうえおあい
35あいうえおあいうえおあいうえおあいうえおあい
36あいうえおあいうえおあいうえおあいうえおあい
37あいうえおあいうえおあいうえおあいうえおあい
38あいうえおあいうえおあいうえおあいうえおあい
39あいうえおあいうえおあいうえおあいうえおあい
40あいうえおあいうえおあいうえおあいうえおあい
41あいうえおあいうえおあいうえおあいうえおあい
42あいうえおあいうえおあいうえおあいうえおあい
43あいうえおあいうえおあいうえおあいうえおあい
44あいうえおあいうえおあいうえおあいうえおあい
45あいうえおあいうえおあいうえおあいうえおあい
46あいうえおあいうえおあいうえおあいうえおあい
47あいうえおあいうえおあいうえおあいうえおあい
48あいうえおあいうえおあいうえおあいうえおあい
49あいうえおあいうえおあいうえおあいうえおあい
50あいうえおあいうえおあいうえおあいうえおあい
\end{document}

